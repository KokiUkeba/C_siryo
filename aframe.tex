\subsection{ポインタと配列(その1)}
\begin{frame}[t, fragile, label=71]
    \frametitle{ポインタと配列}
     配列のポインタを定義することもできる。配列の説明をする際に,
    「メモリ上に同じ型の配列要素が連続して並んだ領域」と書いたが、本当に
    連続しているのか確認してみる。
    \begin{lstlisting}[gobble=8]
        int a[5];
        
        for (int i = 0; i < 5; i++) {
            printf("&a[%d]=%p\n", i, &a[i]);
        }
    \end{lstlisting}
    \begin{block}{出力}
        \verb|&|a[0]=0x7ffeaeae1010\\
        \verb|&|a[1]=0x7ffeaeae1014\\
        \verb|&|a[2]=0x7ffeaeae1018\\
        \verb|&|a[3]=0x7ffeaeae101c\\
        \verb|&|a[4]=0x7ffeaeae1020
    \end{block}
    \begin{textblock*}{0.5\linewidth}(150pt, 190pt)
        ←int型配列なので4[byte]ずつ増加している。
    \end{textblock*}
    \begin{textblock*}{0.3\linewidth}(300pt, 263pt)
        \hyperlink{70}{\beamerbutton{<}}
        \space
        \hyperlink{72}{\beamerbutton{>}}
    \end{textblock*}
\end{frame}

\subsection{ポインタと配列(その2)}
\begin{frame}[t, fragile, label=72]
    \frametitle{ポインタと配列(その2)}
     実は
    \begin{textblock*}{0.3\linewidth}(300pt, 263pt)
        \hyperlink{71}{\beamerbutton{<}}
        \space
        \hyperlink{70}{\beamerbutton{>}}
    \end{textblock*}
\end{frame}
